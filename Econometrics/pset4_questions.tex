% Options for packages loaded elsewhere
\PassOptionsToPackage{unicode}{hyperref}
\PassOptionsToPackage{hyphens}{url}
\PassOptionsToPackage{dvipsnames,svgnames,x11names}{xcolor}
%
\documentclass[
  letterpaper,
  DIV=11,
  numbers=noendperiod]{scrartcl}

\usepackage{amsmath,amssymb}
\usepackage{iftex}
\ifPDFTeX
  \usepackage[T1]{fontenc}
  \usepackage[utf8]{inputenc}
  \usepackage{textcomp} % provide euro and other symbols
\else % if luatex or xetex
  \usepackage{unicode-math}
  \defaultfontfeatures{Scale=MatchLowercase}
  \defaultfontfeatures[\rmfamily]{Ligatures=TeX,Scale=1}
\fi
\usepackage{lmodern}
\ifPDFTeX\else  
    % xetex/luatex font selection
\fi
% Use upquote if available, for straight quotes in verbatim environments
\IfFileExists{upquote.sty}{\usepackage{upquote}}{}
\IfFileExists{microtype.sty}{% use microtype if available
  \usepackage[]{microtype}
  \UseMicrotypeSet[protrusion]{basicmath} % disable protrusion for tt fonts
}{}
\makeatletter
\@ifundefined{KOMAClassName}{% if non-KOMA class
  \IfFileExists{parskip.sty}{%
    \usepackage{parskip}
  }{% else
    \setlength{\parindent}{0pt}
    \setlength{\parskip}{6pt plus 2pt minus 1pt}}
}{% if KOMA class
  \KOMAoptions{parskip=half}}
\makeatother
\usepackage{xcolor}
\setlength{\emergencystretch}{3em} % prevent overfull lines
\setcounter{secnumdepth}{-\maxdimen} % remove section numbering
% Make \paragraph and \subparagraph free-standing
\ifx\paragraph\undefined\else
  \let\oldparagraph\paragraph
  \renewcommand{\paragraph}[1]{\oldparagraph{#1}\mbox{}}
\fi
\ifx\subparagraph\undefined\else
  \let\oldsubparagraph\subparagraph
  \renewcommand{\subparagraph}[1]{\oldsubparagraph{#1}\mbox{}}
\fi

\usepackage{color}
\usepackage{fancyvrb}
\newcommand{\VerbBar}{|}
\newcommand{\VERB}{\Verb[commandchars=\\\{\}]}
\DefineVerbatimEnvironment{Highlighting}{Verbatim}{commandchars=\\\{\}}
% Add ',fontsize=\small' for more characters per line
\usepackage{framed}
\definecolor{shadecolor}{RGB}{241,243,245}
\newenvironment{Shaded}{\begin{snugshade}}{\end{snugshade}}
\newcommand{\AlertTok}[1]{\textcolor[rgb]{0.68,0.00,0.00}{#1}}
\newcommand{\AnnotationTok}[1]{\textcolor[rgb]{0.37,0.37,0.37}{#1}}
\newcommand{\AttributeTok}[1]{\textcolor[rgb]{0.40,0.45,0.13}{#1}}
\newcommand{\BaseNTok}[1]{\textcolor[rgb]{0.68,0.00,0.00}{#1}}
\newcommand{\BuiltInTok}[1]{\textcolor[rgb]{0.00,0.23,0.31}{#1}}
\newcommand{\CharTok}[1]{\textcolor[rgb]{0.13,0.47,0.30}{#1}}
\newcommand{\CommentTok}[1]{\textcolor[rgb]{0.37,0.37,0.37}{#1}}
\newcommand{\CommentVarTok}[1]{\textcolor[rgb]{0.37,0.37,0.37}{\textit{#1}}}
\newcommand{\ConstantTok}[1]{\textcolor[rgb]{0.56,0.35,0.01}{#1}}
\newcommand{\ControlFlowTok}[1]{\textcolor[rgb]{0.00,0.23,0.31}{#1}}
\newcommand{\DataTypeTok}[1]{\textcolor[rgb]{0.68,0.00,0.00}{#1}}
\newcommand{\DecValTok}[1]{\textcolor[rgb]{0.68,0.00,0.00}{#1}}
\newcommand{\DocumentationTok}[1]{\textcolor[rgb]{0.37,0.37,0.37}{\textit{#1}}}
\newcommand{\ErrorTok}[1]{\textcolor[rgb]{0.68,0.00,0.00}{#1}}
\newcommand{\ExtensionTok}[1]{\textcolor[rgb]{0.00,0.23,0.31}{#1}}
\newcommand{\FloatTok}[1]{\textcolor[rgb]{0.68,0.00,0.00}{#1}}
\newcommand{\FunctionTok}[1]{\textcolor[rgb]{0.28,0.35,0.67}{#1}}
\newcommand{\ImportTok}[1]{\textcolor[rgb]{0.00,0.46,0.62}{#1}}
\newcommand{\InformationTok}[1]{\textcolor[rgb]{0.37,0.37,0.37}{#1}}
\newcommand{\KeywordTok}[1]{\textcolor[rgb]{0.00,0.23,0.31}{#1}}
\newcommand{\NormalTok}[1]{\textcolor[rgb]{0.00,0.23,0.31}{#1}}
\newcommand{\OperatorTok}[1]{\textcolor[rgb]{0.37,0.37,0.37}{#1}}
\newcommand{\OtherTok}[1]{\textcolor[rgb]{0.00,0.23,0.31}{#1}}
\newcommand{\PreprocessorTok}[1]{\textcolor[rgb]{0.68,0.00,0.00}{#1}}
\newcommand{\RegionMarkerTok}[1]{\textcolor[rgb]{0.00,0.23,0.31}{#1}}
\newcommand{\SpecialCharTok}[1]{\textcolor[rgb]{0.37,0.37,0.37}{#1}}
\newcommand{\SpecialStringTok}[1]{\textcolor[rgb]{0.13,0.47,0.30}{#1}}
\newcommand{\StringTok}[1]{\textcolor[rgb]{0.13,0.47,0.30}{#1}}
\newcommand{\VariableTok}[1]{\textcolor[rgb]{0.07,0.07,0.07}{#1}}
\newcommand{\VerbatimStringTok}[1]{\textcolor[rgb]{0.13,0.47,0.30}{#1}}
\newcommand{\WarningTok}[1]{\textcolor[rgb]{0.37,0.37,0.37}{\textit{#1}}}

\providecommand{\tightlist}{%
  \setlength{\itemsep}{0pt}\setlength{\parskip}{0pt}}\usepackage{longtable,booktabs,array}
\usepackage{calc} % for calculating minipage widths
% Correct order of tables after \paragraph or \subparagraph
\usepackage{etoolbox}
\makeatletter
\patchcmd\longtable{\par}{\if@noskipsec\mbox{}\fi\par}{}{}
\makeatother
% Allow footnotes in longtable head/foot
\IfFileExists{footnotehyper.sty}{\usepackage{footnotehyper}}{\usepackage{footnote}}
\makesavenoteenv{longtable}
\usepackage{graphicx}
\makeatletter
\def\maxwidth{\ifdim\Gin@nat@width>\linewidth\linewidth\else\Gin@nat@width\fi}
\def\maxheight{\ifdim\Gin@nat@height>\textheight\textheight\else\Gin@nat@height\fi}
\makeatother
% Scale images if necessary, so that they will not overflow the page
% margins by default, and it is still possible to overwrite the defaults
% using explicit options in \includegraphics[width, height, ...]{}
\setkeys{Gin}{width=\maxwidth,height=\maxheight,keepaspectratio}
% Set default figure placement to htbp
\makeatletter
\def\fps@figure{htbp}
\makeatother

\KOMAoption{captions}{tableheading}
\makeatletter
\makeatother
\makeatletter
\makeatother
\makeatletter
\@ifpackageloaded{caption}{}{\usepackage{caption}}
\AtBeginDocument{%
\ifdefined\contentsname
  \renewcommand*\contentsname{Table of contents}
\else
  \newcommand\contentsname{Table of contents}
\fi
\ifdefined\listfigurename
  \renewcommand*\listfigurename{List of Figures}
\else
  \newcommand\listfigurename{List of Figures}
\fi
\ifdefined\listtablename
  \renewcommand*\listtablename{List of Tables}
\else
  \newcommand\listtablename{List of Tables}
\fi
\ifdefined\figurename
  \renewcommand*\figurename{Figure}
\else
  \newcommand\figurename{Figure}
\fi
\ifdefined\tablename
  \renewcommand*\tablename{Table}
\else
  \newcommand\tablename{Table}
\fi
}
\@ifpackageloaded{float}{}{\usepackage{float}}
\floatstyle{ruled}
\@ifundefined{c@chapter}{\newfloat{codelisting}{h}{lop}}{\newfloat{codelisting}{h}{lop}[chapter]}
\floatname{codelisting}{Listing}
\newcommand*\listoflistings{\listof{codelisting}{List of Listings}}
\makeatother
\makeatletter
\@ifpackageloaded{caption}{}{\usepackage{caption}}
\@ifpackageloaded{subcaption}{}{\usepackage{subcaption}}
\makeatother
\makeatletter
\@ifpackageloaded{tcolorbox}{}{\usepackage[skins,breakable]{tcolorbox}}
\makeatother
\makeatletter
\@ifundefined{shadecolor}{\definecolor{shadecolor}{rgb}{.97, .97, .97}}
\makeatother
\makeatletter
\makeatother
\makeatletter
\makeatother
\ifLuaTeX
  \usepackage{selnolig}  % disable illegal ligatures
\fi
\IfFileExists{bookmark.sty}{\usepackage{bookmark}}{\usepackage{hyperref}}
\IfFileExists{xurl.sty}{\usepackage{xurl}}{} % add URL line breaks if available
\urlstyle{same} % disable monospaced font for URLs
\hypersetup{
  pdftitle={Homework 4},
  pdfauthor={Ashutosh Ekade},
  colorlinks=true,
  linkcolor={blue},
  filecolor={Maroon},
  citecolor={Blue},
  urlcolor={Blue},
  pdfcreator={LaTeX via pandoc}}

\title{Homework 4}
\author{Ashutosh Ekade}
\date{}

\begin{document}
\maketitle
\ifdefined\Shaded\renewenvironment{Shaded}{\begin{tcolorbox}[interior hidden, sharp corners, borderline west={3pt}{0pt}{shadecolor}, frame hidden, enhanced, breakable, boxrule=0pt]}{\end{tcolorbox}}\fi

\hypertarget{question-1-linear-regression-with-normal-errors}{%
\subsection{Question 1: Linear Regression with Normal
Errors}\label{question-1-linear-regression-with-normal-errors}}

Load the \texttt{BostonHousing2} dataset from the \texttt{mlbench}
package, which has data on 506 census tracts from the 1970's. Assume the
model is \(Y = \beta_0 + \beta_1X_1 + \beta_2X_2 + \beta_3X_3 + e\) with
\(e \sim N(0,\sigma^2)\), observations are independent, and where \(Y\)
denotes \texttt{medv} (the median value of homes in 1000's of dollars),
\(X_1\) denotes \texttt{rm} (the average number of rooms per home),
\(X_2\) denotes \texttt{age} (the proportion of older homes), and
\(X_3\) denotes \texttt{crim} (the crime rate in the area). See
\texttt{?mlbench::BostonHousing2} for definitions of these variables.

\begin{Shaded}
\begin{Highlighting}[]
\CommentTok{\#install.packages(\textquotesingle{}mlbench\textquotesingle{})}
\FunctionTok{library}\NormalTok{(mlbench)}
\FunctionTok{data}\NormalTok{(}\StringTok{\textquotesingle{}BostonHousing2\textquotesingle{}}\NormalTok{)}
\end{Highlighting}
\end{Shaded}

\hypertarget{a-write-mathematically-the-joint-log-likelihood-function.-what-link-function-or-inverse-link-function-is-required-to-connect-mu_i-to-x_ibeta}{%
\paragraph{\texorpdfstring{1.a) Write, mathematically, the joint
log-likelihood function. What link function (or inverse link function)
is required to ``connect'' \(\mu_i\) to
\(x_i'\beta\)?}{1.a) Write, mathematically, the joint log-likelihood function. What link function (or inverse link function) is required to ``connect'' \textbackslash mu\_i to x\_i\textquotesingle\textbackslash beta?}}\label{a-write-mathematically-the-joint-log-likelihood-function.-what-link-function-or-inverse-link-function-is-required-to-connect-mu_i-to-x_ibeta}}

\$\$

\begin{align}
& \\l(\beta, \sigma^2) 
&= \sum_{i=1}^{n} \log f(Y_i | Y_i, \beta, \sigma^2) \\
&= -\frac{n}{2} \log(2\pi) - \frac{n}{2} \log(\sigma^2) -\frac{1}{2\sigma^2} \sum_{i=1}^{n} (Y_i - X'_i \beta)^2 \\
&= -\frac{n}{2} \log(2\pi) - \frac{n}{2} \log(\sigma^2) -\frac{1}{2\sigma^2} \sum_{i=1}^{n} (Y - X\beta)'(Y - X\beta) \\
\end{align}

\$\$

\hypertarget{b-write-an-r-function-to-calculate-the-log-likelihood-function-from-1a-above.-your-function-should-take-3-arguments-1-theta-beta-sigma2-a-vector-of-all-five-parameters-2-an-n-times-k-matrix-x-and-3-a-vector-or-n-times-1-matrix-y.}{%
\paragraph{\texorpdfstring{1.b) Write an \texttt{R} function to
calculate the log-likelihood function from 1a above. Your function
should take 3 arguments: (1) \(\theta = (\beta, \sigma^2)\) a vector of
all (five) parameters, (2) an \(n \times k\) matrix \(X\), and (3) a
vector or \(n \times 1\) matrix
\(y\).}{1.b) Write an R function to calculate the log-likelihood function from 1a above. Your function should take 3 arguments: (1) \textbackslash theta = (\textbackslash beta, \textbackslash sigma\^{}2) a vector of all (five) parameters, (2) an n \textbackslash times k matrix X, and (3) a vector or n \textbackslash times 1 matrix y.}}\label{b-write-an-r-function-to-calculate-the-log-likelihood-function-from-1a-above.-your-function-should-take-3-arguments-1-theta-beta-sigma2-a-vector-of-all-five-parameters-2-an-n-times-k-matrix-x-and-3-a-vector-or-n-times-1-matrix-y.}}

\begin{Shaded}
\begin{Highlighting}[]
\NormalTok{log\_likelihood }\OtherTok{\textless{}{-}} \ControlFlowTok{function}\NormalTok{(theta, X, y) \{}
\NormalTok{      beta }\OtherTok{\textless{}{-}}\NormalTok{ theta[}\DecValTok{1}\SpecialCharTok{:}\DecValTok{4}\NormalTok{]}
\NormalTok{      sigma\_sq }\OtherTok{\textless{}{-}}\NormalTok{ theta[}\DecValTok{5}\NormalTok{]}
      \CommentTok{\# log{-}likelihood}
\NormalTok{      n }\OtherTok{\textless{}{-}} \FunctionTok{length}\NormalTok{(y)}
\NormalTok{      residuals }\OtherTok{\textless{}{-}}\NormalTok{ y }\SpecialCharTok{{-}}\NormalTok{ X }\SpecialCharTok{\%*\%}\NormalTok{ beta}
\NormalTok{      log\_likelihood\_value }\OtherTok{\textless{}{-}} \SpecialCharTok{{-}}\NormalTok{n}\SpecialCharTok{/}\DecValTok{2} \SpecialCharTok{*} \FunctionTok{log}\NormalTok{(}\DecValTok{2} \SpecialCharTok{*}\NormalTok{ pi) }\SpecialCharTok{{-}}\NormalTok{ n}\SpecialCharTok{/}\DecValTok{2} \SpecialCharTok{*} \FunctionTok{log}\NormalTok{(sigma\_sq) }\SpecialCharTok{{-}} \DecValTok{1}\SpecialCharTok{/}\NormalTok{(}\DecValTok{2} \SpecialCharTok{*}\NormalTok{ sigma\_sq) }\SpecialCharTok{*} \FunctionTok{sum}\NormalTok{(residuals}\SpecialCharTok{\^{}}\DecValTok{2}\NormalTok{)}
    \FunctionTok{return}\NormalTok{(log\_likelihood\_value)}
\NormalTok{  \}}
\end{Highlighting}
\end{Shaded}

\hypertarget{c-use-optim-and-your-function-from-1b-to-find-hatbeta_textmle-and-hatsigma2_textmle-as-well-as-their-standard-errors.-you-may-find-it-helpful-to-initialize-your-search-at-0-for-the-beta-parameters-and-textvary-for-sigma2.-you-will-likely-find-some-differences-in-your-standard-errors-compared-to-the-output-from-lm-in-1d-below.-this-is-because-lm-uses-analytical-expressions-for-the-standard-errors-whereas-optim-uses-numerical-approximations-to-the-hessian-matrix.}{%
\paragraph{\texorpdfstring{1.c) Use \texttt{optim()} and your function
from 1b to find \(\hat{\beta}_\text{MLE}\) and
\(\hat{\sigma}^2_\text{MLE}\) as well as their standard errors. You may
find it helpful to initialize your search at 0 for the \(\beta\)
parameters and \(\text{var}(y)\) for \(\sigma^2\). You will likely find
some differences in your standard errors compared to the output from
\texttt{lm()} in 1d below. This is because \texttt{lm()} uses analytical
expressions for the standard errors, whereas \texttt{optim()} uses
numerical approximations to the Hessian
matrix.}{1.c) Use optim() and your function from 1b to find \textbackslash hat\{\textbackslash beta\}\_\textbackslash text\{MLE\} and \textbackslash hat\{\textbackslash sigma\}\^{}2\_\textbackslash text\{MLE\} as well as their standard errors. You may find it helpful to initialize your search at 0 for the \textbackslash beta parameters and \textbackslash text\{var\}(y) for \textbackslash sigma\^{}2. You will likely find some differences in your standard errors compared to the output from lm() in 1d below. This is because lm() uses analytical expressions for the standard errors, whereas optim() uses numerical approximations to the Hessian matrix.}}\label{c-use-optim-and-your-function-from-1b-to-find-hatbeta_textmle-and-hatsigma2_textmle-as-well-as-their-standard-errors.-you-may-find-it-helpful-to-initialize-your-search-at-0-for-the-beta-parameters-and-textvary-for-sigma2.-you-will-likely-find-some-differences-in-your-standard-errors-compared-to-the-output-from-lm-in-1d-below.-this-is-because-lm-uses-analytical-expressions-for-the-standard-errors-whereas-optim-uses-numerical-approximations-to-the-hessian-matrix.}}

\begin{Shaded}
\begin{Highlighting}[]
\FunctionTok{set.seed}\NormalTok{(}\DecValTok{1234}\NormalTok{)}
\NormalTok{X }\OtherTok{\textless{}{-}} \FunctionTok{cbind}\NormalTok{(}\DecValTok{1}\NormalTok{,BostonHousing2}\SpecialCharTok{$}\NormalTok{rm,BostonHousing2}\SpecialCharTok{$}\NormalTok{age,BostonHousing2}\SpecialCharTok{$}\NormalTok{crim)}
\NormalTok{Y }\OtherTok{\textless{}{-}} \FunctionTok{matrix}\NormalTok{(BostonHousing2}\SpecialCharTok{$}\NormalTok{medv,}\AttributeTok{ncol=}\DecValTok{1}\NormalTok{)}
\NormalTok{initial\_parameters }\OtherTok{\textless{}{-}} \FunctionTok{c}\NormalTok{(}\DecValTok{1}\NormalTok{,}\DecValTok{1}\NormalTok{,}\DecValTok{1}\NormalTok{,}\DecValTok{1}\NormalTok{,}\FunctionTok{var}\NormalTok{(Y))}
\CommentTok{\#optimization function is called}
\NormalTok{result }\OtherTok{\textless{}{-}} \FunctionTok{optim}\NormalTok{(}\AttributeTok{par =}\NormalTok{ initial\_parameters, }\AttributeTok{fn =}\NormalTok{ log\_likelihood,  }\AttributeTok{X=}\NormalTok{X, }\AttributeTok{y=}\NormalTok{Y,}\AttributeTok{control =} \FunctionTok{list}\NormalTok{(}\AttributeTok{fnscale =} \SpecialCharTok{{-}}\DecValTok{1}\NormalTok{),}\AttributeTok{hessian =} \ConstantTok{TRUE}\NormalTok{,}\AttributeTok{method =} \StringTok{"BFGS"}\NormalTok{)}
\end{Highlighting}
\end{Shaded}

\begin{verbatim}
Warning in log(sigma_sq): NaNs produced

Warning in log(sigma_sq): NaNs produced
\end{verbatim}

\begin{Shaded}
\begin{Highlighting}[]
\FunctionTok{print}\NormalTok{(result) }
\end{Highlighting}
\end{Shaded}

\begin{verbatim}
$par
[1] -23.60366982   8.03253919  -0.05224289  -0.21102912  36.83582886

$value
[1] -1630.435

$counts
function gradient 
      58       23 

$convergence
[1] 0

$message
NULL

$hessian
              [,1]          [,2]          [,3]          [,4]          [,5]
[1,] -1.373663e+01 -8.632967e+01 -9.419878e+02 -4.963762e+01 -5.229595e-06
[2,] -8.632967e+01 -5.493184e+02 -5.854902e+03 -2.937887e+02 -8.458301e-05
[3,] -9.419878e+02 -5.854902e+03 -7.545954e+04 -4.574757e+03 -3.916512e-05
[4,] -4.963762e+01 -2.937887e+02 -4.574757e+03 -1.193684e+03 -3.902301e-05
[5,] -5.229595e-06 -8.458301e-05 -3.916512e-05 -3.902301e-05 -1.864794e-01
\end{verbatim}

\begin{Shaded}
\begin{Highlighting}[]
\CommentTok{\#MLE estimates}
\NormalTok{beta\_mle }\OtherTok{\textless{}{-}}\NormalTok{ result}\SpecialCharTok{$}\NormalTok{par[}\DecValTok{1}\SpecialCharTok{:}\DecValTok{4}\NormalTok{]}
\NormalTok{sigma\_square\_mle }\OtherTok{\textless{}{-}}\NormalTok{ result}\SpecialCharTok{$}\NormalTok{par[}\DecValTok{5}\NormalTok{]}
\NormalTok{std\_error }\OtherTok{\textless{}{-}} \FunctionTok{sqrt}\NormalTok{(}\SpecialCharTok{{-}}\FunctionTok{diag}\NormalTok{(}\FunctionTok{solve}\NormalTok{(result}\SpecialCharTok{$}\NormalTok{hessian)))}
\FunctionTok{print}\NormalTok{(}\StringTok{"Beta MLE"}\NormalTok{)}
\end{Highlighting}
\end{Shaded}

\begin{verbatim}
[1] "Beta MLE"
\end{verbatim}

\begin{Shaded}
\begin{Highlighting}[]
\FunctionTok{print}\NormalTok{(beta\_mle)}
\end{Highlighting}
\end{Shaded}

\begin{verbatim}
[1] -23.60366982   8.03253919  -0.05224289  -0.21102912
\end{verbatim}

\begin{Shaded}
\begin{Highlighting}[]
\FunctionTok{print}\NormalTok{(}\StringTok{"Sigma Square MLE"}\NormalTok{)}
\end{Highlighting}
\end{Shaded}

\begin{verbatim}
[1] "Sigma Square MLE"
\end{verbatim}

\begin{Shaded}
\begin{Highlighting}[]
\FunctionTok{print}\NormalTok{(sigma\_square\_mle)}
\end{Highlighting}
\end{Shaded}

\begin{verbatim}
[1] 36.83583
\end{verbatim}

\begin{Shaded}
\begin{Highlighting}[]
\FunctionTok{print}\NormalTok{(}\StringTok{"Standard Error"}\NormalTok{)}
\end{Highlighting}
\end{Shaded}

\begin{verbatim}
[1] "Standard Error"
\end{verbatim}

\begin{Shaded}
\begin{Highlighting}[]
\FunctionTok{print}\NormalTok{(std\_error)}
\end{Highlighting}
\end{Shaded}

\begin{verbatim}
[1] 2.75832954 0.40040222 0.01042100 0.03392967 2.31571187
\end{verbatim}

\hypertarget{d-use-lm-and-summary-to-check-your-work-in-1c-above.}{%
\paragraph{\texorpdfstring{1.d) Use \texttt{lm()} and \texttt{summary()}
to check your work in 1c
above.}{1.d) Use lm() and summary() to check your work in 1c above.}}\label{d-use-lm-and-summary-to-check-your-work-in-1c-above.}}

\begin{Shaded}
\begin{Highlighting}[]
\FunctionTok{summary}\NormalTok{(}\FunctionTok{lm}\NormalTok{(BostonHousing2}\SpecialCharTok{$}\NormalTok{medv }\SpecialCharTok{\textasciitilde{}}\NormalTok{ BostonHousing2}\SpecialCharTok{$}\NormalTok{rm}\SpecialCharTok{+}\NormalTok{BostonHousing2}\SpecialCharTok{$}\NormalTok{age}\SpecialCharTok{+}\NormalTok{BostonHousing2}\SpecialCharTok{$}\NormalTok{crim))}
\end{Highlighting}
\end{Shaded}

\begin{verbatim}

Call:
lm(formula = BostonHousing2$medv ~ BostonHousing2$rm + BostonHousing2$age + 
    BostonHousing2$crim)

Residuals:
    Min      1Q  Median      3Q     Max 
-19.959  -3.143  -0.633   2.150  39.940 

Coefficients:
                     Estimate Std. Error t value Pr(>|t|)    
(Intercept)         -23.60556    2.76938  -8.524  < 2e-16 ***
BostonHousing2$rm     8.03284    0.40201  19.982  < 2e-16 ***
BostonHousing2$age   -0.05224    0.01046  -4.993 8.21e-07 ***
BostonHousing2$crim  -0.21102    0.03407  -6.195 1.22e-09 ***
---
Signif. codes:  0 '***' 0.001 '**' 0.01 '*' 0.05 '.' 0.1 ' ' 1

Residual standard error: 6.094 on 502 degrees of freedom
Multiple R-squared:  0.5636,    Adjusted R-squared:  0.561 
F-statistic: 216.1 on 3 and 502 DF,  p-value: < 2.2e-16
\end{verbatim}

\hypertarget{e-do-the-signs-on-the-3-estimated-slope-coefficients-make-sense-why-or-why-not}{%
\paragraph{1.e) Do the signs on the 3 estimated slope coefficients make
sense? Why or why
not?}\label{e-do-the-signs-on-the-3-estimated-slope-coefficients-make-sense-why-or-why-not}}

The estimated slope coefficients align with expectations for the median
home value determination. As generally observed, the number of rooms per
home shows a positive relationship, while both age and crime rate
exhibit inverse correlations with home values. Older houses typically
depreciate in value over time, reflecting a negative correlation, while
higher crime rates often decrease the desirability of an area, leading
to lower demand and subsequently lower home values. Therefore, the
estimated coefficients in the model are anticipated to be negative,
reflecting the adverse effects of house age and crime rate on the
estimated median home value.

\hypertarget{f-in-1c-optim-returned-the-hessian-matrix-which-enabled-you-to-calculate-the-standard-errors-of-the-mles.-in-1d-lm-provided-the-standard-errors-from-analytical-expressions-we-derived-in-class.-suppose-however-that-you-were-unable-to-calculate-the-standard-errors-and-needed-to-use-a-bootstrap-to-estimate-them.-write-r-code-to-perform-a-bootstrap-to-estimate-the-standard-errors-of-the-3-hatbeta_textmle-parameters.-each-time-through-the-loop-you-may-use-lm-and-coef-if-you-would-like-or-you-may-calculate-hatbeta-directly-from-matrices-you-may-not-use-the-boot-function-from-the-boot-package.}{%
\paragraph{\texorpdfstring{1.f) In 1c, \texttt{optim()} returned the
hessian matrix, which enabled you to calculate the standard errors of
the MLEs. In 1d, \texttt{lm()} provided the standard errors from
analytical expressions we derived in class. Suppose, however, that you
were unable to calculate the standard errors and needed to use a
bootstrap to estimate them. Write \texttt{R} code to perform a bootstrap
to estimate the standard errors of the 3 \(\hat{\beta}_\text{MLE}\)
parameters. Each time through the loop, you may use \texttt{lm()} and
\texttt{coef()} if you would like, or you may calculate \(\hat{\beta}\)
directly from matrices; you may \textbf{not} use the \texttt{boot()}
function from the \texttt{boot}
package.}{1.f) In 1c, optim() returned the hessian matrix, which enabled you to calculate the standard errors of the MLEs. In 1d, lm() provided the standard errors from analytical expressions we derived in class. Suppose, however, that you were unable to calculate the standard errors and needed to use a bootstrap to estimate them. Write R code to perform a bootstrap to estimate the standard errors of the 3 \textbackslash hat\{\textbackslash beta\}\_\textbackslash text\{MLE\} parameters. Each time through the loop, you may use lm() and coef() if you would like, or you may calculate \textbackslash hat\{\textbackslash beta\} directly from matrices; you may not use the boot() function from the boot package.}}\label{f-in-1c-optim-returned-the-hessian-matrix-which-enabled-you-to-calculate-the-standard-errors-of-the-mles.-in-1d-lm-provided-the-standard-errors-from-analytical-expressions-we-derived-in-class.-suppose-however-that-you-were-unable-to-calculate-the-standard-errors-and-needed-to-use-a-bootstrap-to-estimate-them.-write-r-code-to-perform-a-bootstrap-to-estimate-the-standard-errors-of-the-3-hatbeta_textmle-parameters.-each-time-through-the-loop-you-may-use-lm-and-coef-if-you-would-like-or-you-may-calculate-hatbeta-directly-from-matrices-you-may-not-use-the-boot-function-from-the-boot-package.}}

\begin{Shaded}
\begin{Highlighting}[]
\FunctionTok{set.seed}\NormalTok{(}\DecValTok{1234}\NormalTok{)}
\NormalTok{B }\OtherTok{\textless{}{-}} \DecValTok{1000}
\NormalTok{n }\OtherTok{\textless{}{-}} \FunctionTok{nrow}\NormalTok{(X)}
\NormalTok{res }\OtherTok{\textless{}{-}} \FunctionTok{matrix}\NormalTok{(}\ConstantTok{NA\_real\_}\NormalTok{, }\AttributeTok{nrow=}\NormalTok{B, }\AttributeTok{ncol=}\FunctionTok{ncol}\NormalTok{(X))}
\NormalTok{X\_test }\OtherTok{\textless{}{-}}\NormalTok{ X[,}\FunctionTok{c}\NormalTok{(}\DecValTok{2}\NormalTok{,}\DecValTok{3}\NormalTok{,}\DecValTok{4}\NormalTok{)]}

\ControlFlowTok{for}\NormalTok{(b }\ControlFlowTok{in} \DecValTok{1}\SpecialCharTok{:}\NormalTok{B) \{}
\NormalTok{  draws }\OtherTok{\textless{}{-}} \FunctionTok{sample}\NormalTok{(}\DecValTok{1}\SpecialCharTok{:}\NormalTok{n, }\AttributeTok{size=}\NormalTok{n, }\AttributeTok{replace=}\NormalTok{T)}
\NormalTok{  Y\_boot }\OtherTok{\textless{}{-}}\NormalTok{ Y[draws,]}
\NormalTok{  X\_boot }\OtherTok{\textless{}{-}}\NormalTok{ X\_test[draws,]}
\NormalTok{  res[b,] }\OtherTok{\textless{}{-}} \FunctionTok{lm}\NormalTok{(Y\_boot }\SpecialCharTok{\textasciitilde{}}\NormalTok{ X\_boot)}\SpecialCharTok{$}\NormalTok{coef}
\NormalTok{\}}

\NormalTok{serr }\OtherTok{\textless{}{-}} \FunctionTok{apply}\NormalTok{(res, }\DecValTok{2}\NormalTok{, }\ControlFlowTok{function}\NormalTok{(x) }\FunctionTok{sqrt}\NormalTok{(}\FunctionTok{var}\NormalTok{(x)))}

\FunctionTok{print}\NormalTok{(}\StringTok{"The standard errors estimated from the bootstrap are:"}\NormalTok{)}
\end{Highlighting}
\end{Shaded}

\begin{verbatim}
[1] "The standard errors estimated from the bootstrap are:"
\end{verbatim}

\begin{Shaded}
\begin{Highlighting}[]
\NormalTok{serr[}\FunctionTok{cbind}\NormalTok{(}\DecValTok{2}\NormalTok{,}\DecValTok{3}\NormalTok{,}\DecValTok{4}\NormalTok{)]}
\end{Highlighting}
\end{Shaded}

\begin{verbatim}
[1] 0.67701162 0.00927441 0.03502999
\end{verbatim}

\hypertarget{g-use-your-estimates-and-standard-errors-from-1c-above-to-test-whether-the-3-beta-slope-coefficients-are-each-separately-statistically-significantly-different-from-zero-at-a-95-confidence-level.-check-your-results-against-the-output-in-1d-above.}{%
\paragraph{\texorpdfstring{1.g) Use your estimates and standard errors
from 1c above to test whether the 3 \(\beta\) slope coefficients are
each (separately) statistically significantly different from zero at a
95\% confidence level. Check your results against the output in 1d
above.}{1.g) Use your estimates and standard errors from 1c above to test whether the 3 \textbackslash beta slope coefficients are each (separately) statistically significantly different from zero at a 95\% confidence level. Check your results against the output in 1d above.}}\label{g-use-your-estimates-and-standard-errors-from-1c-above-to-test-whether-the-3-beta-slope-coefficients-are-each-separately-statistically-significantly-different-from-zero-at-a-95-confidence-level.-check-your-results-against-the-output-in-1d-above.}}

\begin{Shaded}
\begin{Highlighting}[]
\NormalTok{alpha }\OtherTok{\textless{}{-}} \FloatTok{0.05}
\NormalTok{critical\_value }\OtherTok{\textless{}{-}} \FunctionTok{qt}\NormalTok{(}\DecValTok{1}\SpecialCharTok{{-}}\NormalTok{alpha}\SpecialCharTok{/}\DecValTok{2}\NormalTok{,n}\DecValTok{{-}3}\NormalTok{)}
\NormalTok{beta\_stat }\OtherTok{\textless{}{-}}\NormalTok{ beta\_mle}\SpecialCharTok{/}\NormalTok{std\_error[}\DecValTok{1}\SpecialCharTok{:}\DecValTok{4}\NormalTok{]}

\FunctionTok{print}\NormalTok{(}\StringTok{"Beta Statistics"}\NormalTok{)}
\end{Highlighting}
\end{Shaded}

\begin{verbatim}
[1] "Beta Statistics"
\end{verbatim}

\begin{Shaded}
\begin{Highlighting}[]
\NormalTok{beta\_stat}
\end{Highlighting}
\end{Shaded}

\begin{verbatim}
[1] -8.557233 20.061175 -5.013230 -6.219605
\end{verbatim}

\begin{Shaded}
\begin{Highlighting}[]
\NormalTok{critical\_value }\OtherTok{\textless{}{-}} \FunctionTok{qt}\NormalTok{(}\DecValTok{1} \SpecialCharTok{{-}}\NormalTok{ alpha }\SpecialCharTok{/} \DecValTok{2}\NormalTok{, n}\DecValTok{{-}3}\NormalTok{)}
\FunctionTok{print}\NormalTok{(}\StringTok{"Critical Value"}\NormalTok{)}
\end{Highlighting}
\end{Shaded}

\begin{verbatim}
[1] "Critical Value"
\end{verbatim}

\begin{Shaded}
\begin{Highlighting}[]
\FunctionTok{print}\NormalTok{(critical\_value)}
\end{Highlighting}
\end{Shaded}

\begin{verbatim}
[1] 1.964691
\end{verbatim}

\begin{Shaded}
\begin{Highlighting}[]
\FunctionTok{print}\NormalTok{(}\StringTok{"All the 3 slope coefficients are statistically significant (at 95\% confidence level) because absolute value of each t{-}stat is greater than critical value."}\NormalTok{)}
\end{Highlighting}
\end{Shaded}

\begin{verbatim}
[1] "All the 3 slope coefficients are statistically significant (at 95% confidence level) because absolute value of each t-stat is greater than critical value."
\end{verbatim}

\newpage

\hypertarget{question-2-a-poisson-model-for-count-data}{%
\subsection{Question 2: A Poisson Model for Count
Data}\label{question-2-a-poisson-model-for-count-data}}

Load the \texttt{trading\_behavior} dataset.

The data provides 200 observations on equity trading behavior of
Anderson students. \texttt{id} is an anonymized identifier for the
student. \texttt{numtrades} is the median weekly number trades made by
each student during the Fall quarter. \texttt{program} indicates whether
the student is in the MSBA (1), MBA (2), or MFE (3) program (note that
you may need to store this variable as a factor or convert it to a set
of dummy variables when using it to fit a statistical model).
\texttt{finlittest} is the students' scores on a financial literacy test
taken before entering their graduate program (higher scores indicate
higher financial ``literacy'').

Assume you want to model the number of trades as a function of graduate
program (where \(\mathbbm{1}\) is an indicator function) and financial
literacy:

\[ y_i \sim \text{Pois}(\mu_i) \]

\[ \log \mu_i = \beta_0 + \beta_1\mathbbm{1}(MBA) + \beta_2\mathbbm{1}(MFE) + \beta_3 \text{finlittest} \]

\begin{Shaded}
\begin{Highlighting}[]
\NormalTok{trading\_behavior }\OtherTok{\textless{}{-}} \FunctionTok{read.csv}\NormalTok{(}\StringTok{"trading\_behavior.csv"}\NormalTok{)}
\FunctionTok{head}\NormalTok{(trading\_behavior)}
\end{Highlighting}
\end{Shaded}

\begin{verbatim}
   id numtrades program finlittest
1  45         0     MFE         66
2 108         0    MSBA         66
3  15         0     MFE         69
4  67         0     MFE         67
5 153         0     MFE         65
6  51         0    MSBA         67
\end{verbatim}

\hypertarget{a-a-poisson-density-for-random-variable-y-with-parameter-mu-is-fymu-exp-mumuyy.-suppose-we-let-each-y_i-have-its-own-parameter-mu_i-with-link-function-logcdot-specifically-logmu_i-x_ibeta.-assume-the-data-are-sampled-independently.-write-mathematically-the-joint-log-likelihood-function.}{%
\paragraph{\texorpdfstring{2.a) A Poisson density for random variable
\(Y\) with parameter \(\mu\) is \(f(y|\mu) = \exp(-\mu)\mu^y/y!\).
Suppose we let each \(Y_i\) have it's own parameter \(\mu_i\) with link
function \(\log(\cdot)\): specifically, \(\log(\mu_i) = x_i'\beta\).
Assume the data are sampled independently. Write, mathematically, the
joint log-likelihood
function.}{2.a) A Poisson density for random variable Y with parameter \textbackslash mu is f(y\textbar\textbackslash mu) = \textbackslash exp(-\textbackslash mu)\textbackslash mu\^{}y/y!. Suppose we let each Y\_i have it's own parameter \textbackslash mu\_i with link function \textbackslash log(\textbackslash cdot): specifically, \textbackslash log(\textbackslash mu\_i) = x\_i\textquotesingle\textbackslash beta. Assume the data are sampled independently. Write, mathematically, the joint log-likelihood function.}}\label{a-a-poisson-density-for-random-variable-y-with-parameter-mu-is-fymu-exp-mumuyy.-suppose-we-let-each-y_i-have-its-own-parameter-mu_i-with-link-function-logcdot-specifically-logmu_i-x_ibeta.-assume-the-data-are-sampled-independently.-write-mathematically-the-joint-log-likelihood-function.}}

\$\$

\begin{align}
& f(y|\mu) = \frac{e^{-\mu} \mu^y}{y!} \\
& \log(\mu_i) = \beta_0 + \beta_1 \times 1(\text{MBA}) + \beta_2 \times 1(\text{MFE}) + \beta_3 \times \text{finlittest} \\
& \log f(y_i | \mu_i) = \log \left( \frac{e^{-\mu_i} \mu_i^{y_i}}{y_i!} \right) = -\mu_i + y_i \log(\mu_i) - \log(y_i!) \\
& \ell(\beta) = \sum_{i=1}^{n} \left[ -\mu_i + y_i \log(\mu_i) - \log(y_i!) \right] \\
& \ell(\beta) = \sum_{i=1}^{n} \left[ -\mu_i + y_i \log(\mu_i) - \log(\Gamma(y_i + 1)) \right] \\
& \mu_i = e^{(\beta_0 + \beta_1 \times 1(\text{MBA}_i) + \beta_2 \times 1(\text{MFE}_i) + \beta_3 \times \text{finlittest}_i)}
\end{align}

\$\$

\hypertarget{b-write-an-r-function-to-calculate-the-log-likelihood-function-from-2a-above.-your-function-should-take-3-arguments-1-a-vector-beta-of-all-four-parameters-2-an-n-times-k-matrix-x-and-3-a-vector-or-n-times-1-matrix-y.}{%
\paragraph{\texorpdfstring{2.b) Write an \texttt{R} function to
calculate the log-likelihood function from 2a above. Your function
should take 3 arguments: (1) a vector \(\beta\) of all (four)
parameters, (2) an \(n \times k\) matrix \(X\), and (3) a vector or
\(n \times 1\) matrix
\(y\).}{2.b) Write an R function to calculate the log-likelihood function from 2a above. Your function should take 3 arguments: (1) a vector \textbackslash beta of all (four) parameters, (2) an n \textbackslash times k matrix X, and (3) a vector or n \textbackslash times 1 matrix y.}}\label{b-write-an-r-function-to-calculate-the-log-likelihood-function-from-2a-above.-your-function-should-take-3-arguments-1-a-vector-beta-of-all-four-parameters-2-an-n-times-k-matrix-x-and-3-a-vector-or-n-times-1-matrix-y.}}

\begin{Shaded}
\begin{Highlighting}[]
\NormalTok{trading\_behavior}\SpecialCharTok{$}\NormalTok{program }\OtherTok{\textless{}{-}} \FunctionTok{relevel}\NormalTok{(}\FunctionTok{factor}\NormalTok{(trading\_behavior}\SpecialCharTok{$}\NormalTok{program), }\AttributeTok{ref =} \StringTok{\textquotesingle{}MSBA\textquotesingle{}}\NormalTok{)}

\NormalTok{x }\OtherTok{\textless{}{-}} \FunctionTok{model.matrix}\NormalTok{(}\SpecialCharTok{\textasciitilde{}}\NormalTok{ program }\SpecialCharTok{+}\NormalTok{ finlittest, }\AttributeTok{data =}\NormalTok{ trading\_behavior)}
\NormalTok{y }\OtherTok{\textless{}{-}} \FunctionTok{as.matrix}\NormalTok{(trading\_behavior}\SpecialCharTok{$}\NormalTok{numtrades)}
\NormalTok{initial\_beta }\OtherTok{\textless{}{-}} \FunctionTok{as.vector}\NormalTok{(}\FunctionTok{rep}\NormalTok{(}\DecValTok{0}\NormalTok{, }\FunctionTok{ncol}\NormalTok{(x)))}

\NormalTok{loglik\_poisson }\OtherTok{\textless{}{-}} \ControlFlowTok{function}\NormalTok{(beta, x, y) \{}
    
\NormalTok{    predicted\_values }\OtherTok{\textless{}{-}}\NormalTok{ x }\SpecialCharTok{\%*\%}\NormalTok{ beta}
    
\NormalTok{    mu }\OtherTok{\textless{}{-}} \FunctionTok{exp}\NormalTok{(predicted\_values)}
    
\NormalTok{    llp }\OtherTok{\textless{}{-}} \FunctionTok{sum}\NormalTok{(}\SpecialCharTok{{-}}\NormalTok{mu }\SpecialCharTok{+}\NormalTok{ y }\SpecialCharTok{*}\NormalTok{ predicted\_values }\SpecialCharTok{{-}} \FunctionTok{lgamma}\NormalTok{(y }\SpecialCharTok{+} \DecValTok{1}\NormalTok{))}
    
    \FunctionTok{return}\NormalTok{(llp)}
\NormalTok{    \}}

\FunctionTok{print}\NormalTok{(}\FunctionTok{loglik\_poisson}\NormalTok{(initial\_beta, x, y))}
\end{Highlighting}
\end{Shaded}

\begin{verbatim}
[1] -247.6471
\end{verbatim}

\hypertarget{c-use-optim-and-your-function-from-1b-to-find-hatbeta_textmle-as-well-as-their-standard-errors.}{%
\paragraph{\texorpdfstring{2.c) Use \texttt{optim()} and your function
from 1b to find \(\hat{\beta}_\text{MLE}\) as well as their standard
errors.}{2.c) Use optim() and your function from 1b to find \textbackslash hat\{\textbackslash beta\}\_\textbackslash text\{MLE\} as well as their standard errors.}}\label{c-use-optim-and-your-function-from-1b-to-find-hatbeta_textmle-as-well-as-their-standard-errors.}}

\begin{Shaded}
\begin{Highlighting}[]
\NormalTok{initial\_beta }\OtherTok{\textless{}{-}} \FunctionTok{as.vector}\NormalTok{(}\FunctionTok{rep}\NormalTok{(}\DecValTok{0}\NormalTok{, }\FunctionTok{ncol}\NormalTok{(x)))}

\FunctionTok{set.seed}\NormalTok{(}\DecValTok{1234}\NormalTok{)}
\NormalTok{result }\OtherTok{\textless{}{-}} \FunctionTok{optim}\NormalTok{(}\AttributeTok{par=}\NormalTok{initial\_beta, }
                \AttributeTok{fn=}\NormalTok{loglik\_poisson, }\AttributeTok{x=}\NormalTok{x, }
                \AttributeTok{y=}\NormalTok{y, }\AttributeTok{method =} \StringTok{"BFGS"}\NormalTok{, }
                \AttributeTok{control=}\FunctionTok{list}\NormalTok{(}\AttributeTok{fnscale=}\SpecialCharTok{{-}}\DecValTok{1}\NormalTok{),}
                \AttributeTok{hessian =} \ConstantTok{TRUE}\NormalTok{)}

\NormalTok{beta\_mle\_poisson }\OtherTok{\textless{}{-}}\NormalTok{ result}\SpecialCharTok{$}\NormalTok{par}
\NormalTok{std\_errors\_mle\_poisson }\OtherTok{\textless{}{-}} \FunctionTok{sqrt}\NormalTok{(}\SpecialCharTok{{-}}\FunctionTok{diag}\NormalTok{(}\FunctionTok{solve}\NormalTok{(result}\SpecialCharTok{$}\NormalTok{hessian)))}

\FunctionTok{print}\NormalTok{(beta\_mle\_poisson)}
\end{Highlighting}
\end{Shaded}

\begin{verbatim}
[1] -6.8385058  1.0859384  0.3604895  0.0682507
\end{verbatim}

\begin{Shaded}
\begin{Highlighting}[]
\FunctionTok{print}\NormalTok{(std\_errors\_mle\_poisson)}
\end{Highlighting}
\end{Shaded}

\begin{verbatim}
[1] 0.8917300 0.3560795 0.4389272 0.0105377
\end{verbatim}

\hypertarget{d-fit-the-model-using-glm.-compare-your-results-to-the-output-from-optim-in-question-2c-above.}{%
\paragraph{\texorpdfstring{2.d) Fit the model using \texttt{glm()}.
Compare your results to the output from \texttt{optim} in question 2c
above.}{2.d) Fit the model using glm(). Compare your results to the output from optim in question 2c above.}}\label{d-fit-the-model-using-glm.-compare-your-results-to-the-output-from-optim-in-question-2c-above.}}

\begin{Shaded}
\begin{Highlighting}[]
\FunctionTok{glm}\NormalTok{(y }\SpecialCharTok{\textasciitilde{}}\NormalTok{ x }\SpecialCharTok{{-}} \DecValTok{1}\NormalTok{, }\AttributeTok{data=}\NormalTok{trading\_behavior, }\AttributeTok{family =} \FunctionTok{poisson}\NormalTok{())}
\end{Highlighting}
\end{Shaded}

\begin{verbatim}

Call:  glm(formula = y ~ x - 1, family = poisson(), data = trading_behavior)

Coefficients:
x(Intercept)   xprogramMBA   xprogramMFE   xfinlittest  
    -7.00093       1.08386       0.36981       0.07015  

Degrees of Freedom: 200 Total (i.e. Null);  196 Residual
Null Deviance:      319.2 
Residual Deviance: 189.4    AIC: 373.5
\end{verbatim}

\hypertarget{e-the-analog-to-the-f-test-from-linear-regression-is-the-likelihood-ratio-test.-the-likelihood-ratio-test-statistic-is-calculated-as}{%
\paragraph{2.e) The ``analog'' to the F-test from linear regression is
the Likelihood Ratio Test. The Likelihood Ratio test statistic is
calculated
as:}\label{e-the-analog-to-the-f-test-from-linear-regression-is-the-likelihood-ratio-test.-the-likelihood-ratio-test-statistic-is-calculated-as}}

\[ LR_n = 2 \times [ \ell_n(\hat{\theta}) - \ell_n(\tilde{\theta})] \]

\hypertarget{where-ell_ncdot-is-the-log-likelihood-function-hattheta-is-the-mle-and-tildetheta-is-a-constrained-parameter-vector-e.g.-suppose-a-null-hypothesis-is-that-theta_20-theta_30.-the-likelihood-ratio-test-statistic-lr_n-has-an-asymptotic-chi-squared-distribution-with-k-degrees-of-freedom-i.e.-chi2_k-where-k-is-the-length-of-the-theta-vector.}{%
\paragraph{\texorpdfstring{where \(\ell_n(\cdot)\) is the log likelihood
function, \(\hat{\theta}\) is the MLE, and \(\tilde{\theta}\) is a
constrained parameter vector (e.g., suppose a Null Hypothesis is that
\(\theta_2=0\) \& \(\theta_3=0\)). The Likelihood Ratio test statistic
\(LR_n\) has an asymptotic chi-squared distribution with \(k\) degrees
of freedom (i.e., \(\chi^2_k\) where \(k\) is the length of the
\(\theta\)
vector).}{where \textbackslash ell\_n(\textbackslash cdot) is the log likelihood function, \textbackslash hat\{\textbackslash theta\} is the MLE, and \textbackslash tilde\{\textbackslash theta\} is a constrained parameter vector (e.g., suppose a Null Hypothesis is that \textbackslash theta\_2=0 \& \textbackslash theta\_3=0). The Likelihood Ratio test statistic LR\_n has an asymptotic chi-squared distribution with k degrees of freedom (i.e., \textbackslash chi\^{}2\_k where k is the length of the \textbackslash theta vector).}}\label{where-ell_ncdot-is-the-log-likelihood-function-hattheta-is-the-mle-and-tildetheta-is-a-constrained-parameter-vector-e.g.-suppose-a-null-hypothesis-is-that-theta_20-theta_30.-the-likelihood-ratio-test-statistic-lr_n-has-an-asymptotic-chi-squared-distribution-with-k-degrees-of-freedom-i.e.-chi2_k-where-k-is-the-length-of-the-theta-vector.}}

\hypertarget{test-the-joint-hypothesis-that-beta_20-beta_30-at-the-95-confidence-level-using-a-likelihood-ratio-test.-specifically-use-your-log-likelihood-function-from-2b-above-and-your-parameter-estimates-from-2c-above-to-calculate-ell_nhatbeta_textmle.-then-replace-beta_2-and-beta_3-with-their-hypothesized-values-and-re-calculate-the-log-likehood-ie-ell_nhatbeta_100hatbeta_4.-next-compute-lr_n-and-compare-the-value-to-the-cut-off-of-a-chi-squared-distribution-with-4-degrees-of-freedom-to-assess-whether-or-not-you-reject-the-null-hypothesis.}{%
\paragraph{\texorpdfstring{Test the joint hypothesis that \(\beta_2=0\)
\& \(\beta_3=0\) at the 95\% confidence level using a Likelihood Ratio
test. Specifically, use your log-likelihood function from 2b above and
your parameter estimates from 2c above to calculate
\(\ell_n(\hat{\beta_\text{MLE}})\). Then replace \(\beta_2\) and
\(\beta_3\) with their hypothesized values and re-calculate the
log-likehood (ie, \(\ell_n([\hat{\beta}_1,0,0,\hat{\beta}_4])\). Next,
compute \(LR_n\) and compare the value to the cut-off of a chi-squared
distribution with 4 degrees of freedom to assess whether or not you
reject the Null
Hypothesis.}{Test the joint hypothesis that \textbackslash beta\_2=0 \& \textbackslash beta\_3=0 at the 95\% confidence level using a Likelihood Ratio test. Specifically, use your log-likelihood function from 2b above and your parameter estimates from 2c above to calculate \textbackslash ell\_n(\textbackslash hat\{\textbackslash beta\_\textbackslash text\{MLE\}\}). Then replace \textbackslash beta\_2 and \textbackslash beta\_3 with their hypothesized values and re-calculate the log-likehood (ie, \textbackslash ell\_n({[}\textbackslash hat\{\textbackslash beta\}\_1,0,0,\textbackslash hat\{\textbackslash beta\}\_4{]}). Next, compute LR\_n and compare the value to the cut-off of a chi-squared distribution with 4 degrees of freedom to assess whether or not you reject the Null Hypothesis.}}\label{test-the-joint-hypothesis-that-beta_20-beta_30-at-the-95-confidence-level-using-a-likelihood-ratio-test.-specifically-use-your-log-likelihood-function-from-2b-above-and-your-parameter-estimates-from-2c-above-to-calculate-ell_nhatbeta_textmle.-then-replace-beta_2-and-beta_3-with-their-hypothesized-values-and-re-calculate-the-log-likehood-ie-ell_nhatbeta_100hatbeta_4.-next-compute-lr_n-and-compare-the-value-to-the-cut-off-of-a-chi-squared-distribution-with-4-degrees-of-freedom-to-assess-whether-or-not-you-reject-the-null-hypothesis.}}

\begin{Shaded}
\begin{Highlighting}[]
\NormalTok{loglik\_mle }\OtherTok{\textless{}{-}} \FunctionTok{loglik\_poisson}\NormalTok{(beta\_mle\_poisson, x, y)}

\NormalTok{beta\_constrained }\OtherTok{\textless{}{-}}\NormalTok{ beta\_mle\_poisson}
\NormalTok{beta\_constrained[}\DecValTok{2}\SpecialCharTok{:}\DecValTok{3}\NormalTok{] }\OtherTok{\textless{}{-}} \DecValTok{0}

\NormalTok{loglik\_constrained }\OtherTok{\textless{}{-}} \FunctionTok{loglik\_poisson}\NormalTok{(beta\_constrained, x, y)}

\CommentTok{\# Log likelihood ratio test statistic calculation}
\NormalTok{LR\_t }\OtherTok{\textless{}{-}} \DecValTok{2} \SpecialCharTok{*}\NormalTok{ (loglik\_mle }\SpecialCharTok{{-}}\NormalTok{ loglik\_constrained)}

\CommentTok{\# Degrees of freedom = number of constrained parameters}
\NormalTok{df }\OtherTok{\textless{}{-}} \DecValTok{2}

\CommentTok{\# Critical value from chi{-}squared distribution}
\NormalTok{chi\_squared\_critical }\OtherTok{\textless{}{-}} \FunctionTok{qchisq}\NormalTok{(}\FloatTok{0.95}\NormalTok{, df)}

\FunctionTok{cat}\NormalTok{(}\StringTok{"Likelihood Ratio Test Statistic:"}\NormalTok{, LR\_t, }\StringTok{"}\SpecialCharTok{\textbackslash{}n}\StringTok{"}\NormalTok{)}
\end{Highlighting}
\end{Shaded}

\begin{verbatim}
Likelihood Ratio Test Statistic: 90.36276 
\end{verbatim}

\begin{Shaded}
\begin{Highlighting}[]
\FunctionTok{cat}\NormalTok{(}\StringTok{"Chi{-}squared critical value at 95\% confidence level:"}\NormalTok{, chi\_squared\_critical, }\StringTok{"}\SpecialCharTok{\textbackslash{}n}\StringTok{"}\NormalTok{)}
\end{Highlighting}
\end{Shaded}

\begin{verbatim}
Chi-squared critical value at 95% confidence level: 5.991465 
\end{verbatim}

\begin{Shaded}
\begin{Highlighting}[]
\CommentTok{\# Decision}
\ControlFlowTok{if}\NormalTok{ (}\FunctionTok{abs}\NormalTok{(LR\_t) }\SpecialCharTok{\textgreater{}}\NormalTok{ chi\_squared\_critical) \{}
  \FunctionTok{cat}\NormalTok{(}\StringTok{"Reject the null hypothesis: beta\_2 and beta\_3 is non{-}zero."}\NormalTok{)}
\NormalTok{  \} }\ControlFlowTok{else}\NormalTok{ \{}
  \FunctionTok{cat}\NormalTok{(}\StringTok{"Fail to reject the null hypothesis: beta\_2 and beta\_3 may both be zero."}\NormalTok{)}
\NormalTok{    \}}
\end{Highlighting}
\end{Shaded}

\begin{verbatim}
Reject the null hypothesis: beta_2 and beta_3 is non-zero.
\end{verbatim}

P-value of LR is 0 and that of chi-square is 0.05. So, we reject the
Null hypothesis.

\newpage

\textbf{Question 3 is OPTIONAL. If you accurately complete it, you will
receive 2 bonus points toward your final grade.}

\hypertarget{question-3-estimating-demand-via-the-multi-nomial-logit-model-mnl}{%
\subsection{Question 3: Estimating Demand via the Multi-Nomial Logit
Model
(MNL)}\label{question-3-estimating-demand-via-the-multi-nomial-logit-model-mnl}}

Suppose you have \(i=1,\ldots,n\) consumers who each select exactly one
product \(j\) from a set of \(J\) products. The outcome variable is the
identity of the product chosen \(y_i \in \{1, \ldots, J\}\) or
equivalently a vector of \(J-1\) zeros and \(1\) one, where the \(1\)
indicates the selected product. For example, if the third product was
chosen out of 4 products, then either \(y=3\) or \(y=(0,0,1,0)\)
depending on how you want to represent it. Suppose also that you have a
vector of data on each product \(x_j\) (eg, size, price, etc.).

The MNL model posits that the probability that consumer \(i\) chooses
product \(j\) is:

\[ \mathbb{P}_i(j) = \frac{e^{x_j'\beta}}{\sum_{k=1}^Je^{x_k'\beta}} \]

For example, if there are 4 products, the probability that consumer
\(i\) chooses product 3 is:

\[ \mathbb{P}_i(3) = \frac{e^{x_3'\beta}}{e^{x_1'\beta} + e^{x_2'\beta} + e^{x_3'\beta} + e^{x_4'\beta}} \]

A clever way to write the individual likelihood function for consumer
\(i\) is the product of the \(J\) probabilities, each raised to the
power of an indicator variable (\(\delta_{ij}\)) that indicates the
chosen product:

\[ L_i(\beta) = \prod_{j=1}^J \mathbb{P}_i(j)^{\delta_{ij}} = \mathbb{P}_i(1)^{\delta_{i1}} \times \ldots \times \mathbb{P}_i(J)^{\delta_{iJ}}\]

Notice that if the consumer selected product \(j=3\), then
\(\delta_{i3}=1\) while \(\delta_{i1}=\delta_{i2}=\delta_{i4}=0\) and
the likelihood is:

\[ L_i(\beta) = \mathbb{P}_i(1)^0 \times \mathbb{P}_i(2)^0 \times \mathbb{P}_i(3)^1 \times \mathbb{P}_i(4)^0 = \mathbb{P}_i(3) = \frac{e^{x_3'\beta}}{\sum_{k=1}^Je^{x_k'\beta}} \]

The joint likelihood (across all consumers) is the product of the \(n\)
individual likelihoods:

\[ L_n(\beta) = \prod_{i=1}^n L_i(\beta) = \prod_{i=1}^n \prod_{j=1}^J \mathbb{P}_i(j)^{\delta_{ij}} \]

And the joint log-likelihood function is:

\[ \ell_n(\beta) = \sum_{i=1}^n \sum_{j=1}^J \delta_{ij} \log(\mathbb{P}_i(j)) \]

Use the \texttt{yogurt\_data} dataset, which provides the anonymized
consumer identifiers (\texttt{id}), a vector indicating the chosen
product (\texttt{y1}:\texttt{y4}), a vector indicating if any products
were ``featured'' in the store as a form of advertising
(\texttt{f1}:\texttt{f4}), and the products' prices
(\texttt{p1}:\texttt{p4}). For example, consumer 1 purchased yogurt 4 at
a price of 0.079/oz and none of the yogurts were featured/advertised at
the time of consumer 1's purchase. Consumers 2 through 7 each bought
yogurt 2, etc.

Let the vector of product features include brand dummy variables for
yogurts 1-3 (omit a dummy for product 4 to avoid multi-collinearity), a
dummy variable to indicate featured, and a continuous variable for
price:

\[ x_j' = [\mathbbm{1}(\text{Yogurt 1}), \mathbbm{1}(\text{Yogurt 2}), \mathbbm{1}(\text{Yogurt 3}), X_f, X_p] \]

You will need to create the product dummies. The variables for featured
and price are included in the dataset. The ``hard part'' of this
likelihood function is organizing the data.

Your task: Code up the log-likelihood function. Use \texttt{optim()} to
find the MLEs for the 5 parameters
(\(\beta_1, \beta_2, \beta_3, \beta_f, \beta_p\)).

(Hint: you should find 2 positive and 1 negative product intercepts, a
small positive coefficient estimate for featured, and a large negative
coefficient estimate for price.)

\begin{Shaded}
\begin{Highlighting}[]
\FunctionTok{library}\NormalTok{(tidyverse)}
\end{Highlighting}
\end{Shaded}

\begin{verbatim}
-- Attaching core tidyverse packages ------------------------ tidyverse 2.0.0 --
v dplyr     1.1.3     v readr     2.1.4
v forcats   1.0.0     v stringr   1.5.0
v ggplot2   3.4.3     v tibble    3.2.1
v lubridate 1.9.2     v tidyr     1.3.0
v purrr     1.0.2     
-- Conflicts ------------------------------------------ tidyverse_conflicts() --
x dplyr::filter() masks stats::filter()
x dplyr::lag()    masks stats::lag()
i Use the conflicted package (<http://conflicted.r-lib.org/>) to force all conflicts to become errors
\end{verbatim}

\begin{Shaded}
\begin{Highlighting}[]
\NormalTok{yogurt\_data }\OtherTok{\textless{}{-}} \FunctionTok{read.csv}\NormalTok{(}\StringTok{"yogurt\_data.csv"}\NormalTok{)}
\NormalTok{yogurt\_data }\OtherTok{\textless{}{-}}\NormalTok{ yogurt\_data }\SpecialCharTok{\%\textgreater{}\%} 
  \FunctionTok{mutate}\NormalTok{(}
    \AttributeTok{yogurt\_1 =} \FunctionTok{as.integer}\NormalTok{(y1 }\SpecialCharTok{==} \DecValTok{1}\NormalTok{),}
    \AttributeTok{yogurt\_2 =} \FunctionTok{as.integer}\NormalTok{(y2 }\SpecialCharTok{==} \DecValTok{1}\NormalTok{),}
    \AttributeTok{yogurt\_3 =} \FunctionTok{as.integer}\NormalTok{(y3 }\SpecialCharTok{==} \DecValTok{1}\NormalTok{)}
\NormalTok{  )}
\NormalTok{log\_likelihood }\OtherTok{\textless{}{-}} \ControlFlowTok{function}\NormalTok{(beta, data) \{}
\NormalTok{  x1Beta }\OtherTok{\textless{}{-}}\NormalTok{ beta[}\DecValTok{1}\NormalTok{] }\SpecialCharTok{+}\NormalTok{ beta[}\DecValTok{4}\NormalTok{] }\SpecialCharTok{*}\NormalTok{ data}\SpecialCharTok{$}\NormalTok{f1 }\SpecialCharTok{+}\NormalTok{ beta[}\DecValTok{5}\NormalTok{] }\SpecialCharTok{*}\NormalTok{ data}\SpecialCharTok{$}\NormalTok{p1}
\NormalTok{  x2Beta }\OtherTok{\textless{}{-}}\NormalTok{ beta[}\DecValTok{2}\NormalTok{] }\SpecialCharTok{+}\NormalTok{ beta[}\DecValTok{4}\NormalTok{] }\SpecialCharTok{*}\NormalTok{ data}\SpecialCharTok{$}\NormalTok{f2 }\SpecialCharTok{+}\NormalTok{ beta[}\DecValTok{5}\NormalTok{] }\SpecialCharTok{*}\NormalTok{ data}\SpecialCharTok{$}\NormalTok{p2}
\NormalTok{  x3Beta }\OtherTok{\textless{}{-}}\NormalTok{ beta[}\DecValTok{3}\NormalTok{] }\SpecialCharTok{+}\NormalTok{ beta[}\DecValTok{4}\NormalTok{] }\SpecialCharTok{*}\NormalTok{ data}\SpecialCharTok{$}\NormalTok{f3 }\SpecialCharTok{+}\NormalTok{ beta[}\DecValTok{5}\NormalTok{] }\SpecialCharTok{*}\NormalTok{ data}\SpecialCharTok{$}\NormalTok{p3}
\NormalTok{  x4Beta }\OtherTok{\textless{}{-}} \DecValTok{0} \SpecialCharTok{+}\NormalTok{ beta[}\DecValTok{4}\NormalTok{] }\SpecialCharTok{*}\NormalTok{ data}\SpecialCharTok{$}\NormalTok{f4 }\SpecialCharTok{+}\NormalTok{ beta[}\DecValTok{5}\NormalTok{] }\SpecialCharTok{*}\NormalTok{ data}\SpecialCharTok{$}\NormalTok{p4}
\NormalTok{  prob\_1 }\OtherTok{\textless{}{-}} \FunctionTok{exp}\NormalTok{(x1Beta) }\SpecialCharTok{/}\NormalTok{ (}\FunctionTok{exp}\NormalTok{(x1Beta) }\SpecialCharTok{+} \FunctionTok{exp}\NormalTok{(x2Beta) }\SpecialCharTok{+} \FunctionTok{exp}\NormalTok{(x3Beta) }\SpecialCharTok{+} \FunctionTok{exp}\NormalTok{(x4Beta))}
\NormalTok{  prob\_2 }\OtherTok{\textless{}{-}} \FunctionTok{exp}\NormalTok{(x2Beta) }\SpecialCharTok{/}\NormalTok{ (}\FunctionTok{exp}\NormalTok{(x1Beta) }\SpecialCharTok{+} \FunctionTok{exp}\NormalTok{(x2Beta) }\SpecialCharTok{+} \FunctionTok{exp}\NormalTok{(x3Beta) }\SpecialCharTok{+} \FunctionTok{exp}\NormalTok{(x4Beta))}
\NormalTok{  prob\_3 }\OtherTok{\textless{}{-}} \FunctionTok{exp}\NormalTok{(x3Beta) }\SpecialCharTok{/}\NormalTok{ (}\FunctionTok{exp}\NormalTok{(x1Beta) }\SpecialCharTok{+} \FunctionTok{exp}\NormalTok{(x2Beta) }\SpecialCharTok{+} \FunctionTok{exp}\NormalTok{(x3Beta) }\SpecialCharTok{+} \FunctionTok{exp}\NormalTok{(x4Beta))}
\NormalTok{  prob\_4 }\OtherTok{\textless{}{-}} \FunctionTok{exp}\NormalTok{(x4Beta) }\SpecialCharTok{/}\NormalTok{ (}\FunctionTok{exp}\NormalTok{(x1Beta) }\SpecialCharTok{+} \FunctionTok{exp}\NormalTok{(x2Beta) }\SpecialCharTok{+} \FunctionTok{exp}\NormalTok{(x3Beta) }\SpecialCharTok{+} \FunctionTok{exp}\NormalTok{(x4Beta))}
\NormalTok{  ll }\OtherTok{\textless{}{-}} \FunctionTok{sum}\NormalTok{(data}\SpecialCharTok{$}\NormalTok{yogurt\_1 }\SpecialCharTok{*} \FunctionTok{log}\NormalTok{(prob\_1) }\SpecialCharTok{+}
\NormalTok{              data}\SpecialCharTok{$}\NormalTok{yogurt\_2 }\SpecialCharTok{*} \FunctionTok{log}\NormalTok{(prob\_2) }\SpecialCharTok{+}
\NormalTok{              data}\SpecialCharTok{$}\NormalTok{yogurt\_3 }\SpecialCharTok{*} \FunctionTok{log}\NormalTok{(prob\_3) }\SpecialCharTok{+}
\NormalTok{              (}\DecValTok{1} \SpecialCharTok{{-}}\NormalTok{ data}\SpecialCharTok{$}\NormalTok{yogurt\_1 }\SpecialCharTok{{-}}\NormalTok{ data}\SpecialCharTok{$}\NormalTok{yogurt\_2 }\SpecialCharTok{{-}}\NormalTok{ data}\SpecialCharTok{$}\NormalTok{yogurt\_3) }\SpecialCharTok{*} \FunctionTok{log}\NormalTok{(prob\_4))}
  \FunctionTok{return}\NormalTok{(ll)}
\NormalTok{\} }
\NormalTok{out\_optim }\OtherTok{\textless{}{-}} \FunctionTok{optim}\NormalTok{(}\AttributeTok{par =} \FunctionTok{rep}\NormalTok{(}\DecValTok{0}\NormalTok{, }\DecValTok{5}\NormalTok{), }\AttributeTok{fn =}\NormalTok{ log\_likelihood, }\AttributeTok{data =}\NormalTok{ yogurt\_data, }\AttributeTok{control =} \FunctionTok{list}\NormalTok{(}\AttributeTok{fnscale =} \SpecialCharTok{{-}}\DecValTok{1}\NormalTok{),}\AttributeTok{hessian =} \ConstantTok{TRUE}\NormalTok{, }\AttributeTok{method=}\StringTok{"BFGS"}\NormalTok{)}

\FunctionTok{cat}\NormalTok{(}\StringTok{"Beta1 is"}\NormalTok{, beta\_mle[}\DecValTok{1}\NormalTok{] ,}\StringTok{"}\SpecialCharTok{\textbackslash{}n}\StringTok{"}\NormalTok{, }\StringTok{"Beta2 is"}\NormalTok{, beta\_mle[}\DecValTok{2}\NormalTok{],}\StringTok{"}\SpecialCharTok{\textbackslash{}n}\StringTok{"}\NormalTok{, }\StringTok{"Beta3 is"}\NormalTok{, beta\_mle[}\DecValTok{3}\NormalTok{])}
\end{Highlighting}
\end{Shaded}

\begin{verbatim}
Beta1 is -23.60367 
 Beta2 is 8.032539 
 Beta3 is -0.05224289
\end{verbatim}

\begin{Shaded}
\begin{Highlighting}[]
\NormalTok{out\_optim}\SpecialCharTok{$}\NormalTok{par}
\end{Highlighting}
\end{Shaded}

\begin{verbatim}
[1]   1.3908347   0.6440157  -3.0887416   0.4861072 -37.1591372
\end{verbatim}



\end{document}
